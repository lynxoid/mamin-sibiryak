\documentclass[b5paper,12pt,openany]{book}

\usepackage{fullpage}

\renewcommand{\thesection}{\Roman{section}}

\begin{document}

\title{Stories and Tales}
\author{D. N. Mamin-Sibiryak \\ Translated by D. S. Filippova}

\maketitle

\chapter{Yemelya the Hunter}

\section{}

In a far away land, in the northern parts of the Ural mountains, there sits a small village of Tichki, hidden in the very heart of the impenetrable wild woods. There are only a total of eleven log houses, or ten rather, since the eleventh house stands apart from others, right next to the woods.
% The evegreen pine woods [coniferous] surround the village like a jagged [toothed, crenellated] wall.
The evergreen pine woods stand tall round the village like a jagged castle wall.
Several mountains peak from behind the spruces and firs, their huge gray-blue masses  surrounding Tichki on all sides as if deliberately. The hunchbacked Spring Mountain is the closest to Tichki with its silver, bushy top; its peak is often completely hidden in the thick, gray clouds on gloomy day. Many springs and creeks run down the slopes of Spring Mountain. One such brook happily treads towards Tichki and provides icy cold, clean --- like a tear --- water to all year round.

The houses in Tichki were built without any plan, wherever anyone wanted. Two houses are right above the creek, another --- on the steep side of a mound, and the rest scattered along the river like roaming sheep. There is not even a main street in Tichki, but only a beaten path curving in between the log huts. But, mostly likely, no one in Tichki would need a road anyway, because no one would ride it: there was not a single cart in Tichki. In the summer, this village is usually surrounded by impenetrable marshes, bogs and forest slums [лесная трущоба], so one could hardly reach the village by foot by following the very narrow forest paths, and even then not always. During the bad weather [ненастье], the mountain creeks swell and flood the paths, and it happens often then Tichki hunters have to wait up to three days in a rows for the water to receed.

% zealous, mettlesome, arrant, hardcore, true, hardened
% [записной]
All men in Tichki are arrant hunters. They almost never leave the woods during the summer and winter,
% and it is good that the woods are not too far
and thank god to its close proximity. Every season has its own catch: in the winter you get bear, martens, wolves, foxes; in the fall --- squirells, in the springtime --- wild goats; in the summer --- various wildfowl. To say it shortly, you are at a hard and often dangerous job year round.

In the house by the very woods lives an old hunter Yemelya with his little grandson Greeshootka. Yemelya's house has grown into the ground and now looks at the rest of the world with but a single window; its roof had long rotted through, the bricks from the crumbled chimney scattered around. There was nothing to go by at Yemelya's yard --- no fence, no gate, not even a shed. The only thing was his dog Lisko who would wail in hunger at nights from under the porch made of unhewn logs --- one of the best hunting dogs in Tichki. Yemeleya would starve the unfortunate Lisko three days before any hunt, so that he would scour the area  and track the game better.

``Gramps'!.. Eh, gramps'!'' asked once the little Greeshootka. "Are the does now with the fawns?"

``With the fawns, Greeshootka'' -- responded Yemelya while finishing the braiding on the new basts.

``Say, gramps, we get a fawn\ldots Eh?''

``Wait, we'll get'em\ldots The heat is here now, the does with fawns will hide in the deep woods to keep away from gadflies, that's when I will get you a fawn, Greeshook!''

The boy replied nothing, and only released a heavy sigh. Greeshootka was only six, and now he was abed for a second month straight on a wide wooden bench under a warm deer skin. The boy caught a cold back in spring, when the snow was melting, and still would not improve. His olive skin grew paler and his face looked drawn, his eyes grew bigger, his nose sharper. Yemelya saw that his grandson was wasting away by an hour, but did not know what to do with the trouble. He gave herbs, twice took him to the steam room --- nothing was helping. The boy ate almost nothing. He would sometimes chew on some dark bread, and that's it. There was some salted goat meat left over since the spring, but Greeshook would not even look at it.

``Huh, what the boy wants: a fawn\ldots'', thought old Yemelya to himself, while picking at his bast. "So I must get it."

Yemelya must have been seventy: gray-haired, bent at the back, thin, with long arms. The fingers on Yemelya's hands could barely bend anymore, as if they were wooden sticks. But he still had a sprightly step and would get game whlie hunting. The only thing were his eyes that started to falter, especially during the winters, making it had to see when the snow sparkles and shines all around like diamond dust. It is because of his eyes that the chimney collapsed, and the roof gone rotten, and he himself would sit in the hut when others would be out in the woods.

It was time for the old man to retire, to go up on a stove [печь], but there were no one to take his place, and now Greeshootka was in his charge, he had to take care of him while he could\ldots Greeshootka's father did three years ago of burning ague, mother was snatched by wolves when she was coming back from the village to her hut one winter evening. The child escaped the wolves by mere luck. The mother, wolves gnawing on her legs, covered the child with her body, and Greeshootka remained alive.

The old man had to take in the grandson, and now in addition the boy was in ill health. Misfotunes never come singly\ldots

\section{}

There were the last days of the month of June, the hottest time of the year in Tichki. Only the old and the feeble stayed home. The hunters were all long gone to the woods for the deer. in Yemelya's house, poor Lisko was howling in hunger for the third day straight, just like wolves would in the winter.

``Must be that Yemelya is going hunting,'' would say women in the village.

And it was so. Indeed, Yemelya soon came out of his hut with a flintlock rifle in his arms, untied Lisko and headed for the woods. He was wearing the new basts, a sack with bread on his shoulders, torn kaftan and a warm cap made of deer fur on his head. The old man had long stopped wearing a proper hat, instead, he would wear his deer cap everywhere through summers and winters, which would protect his bald head from the winter cold and the summer swelter [heat] alike.

``Well, Greeshook, get better while I am gone,'' said Yemelya to grandson when leaving. "The old woman Malanya will look after you while I am gone to get a fawn."

``Will you get the fawn, gramps, eh?''

``I said I will.''

``All yellow?''

``All yellow\ldots''

``Well, I will be waiting for you\ldots Just do not miss when you shoot\ldots''

Yemelya was long ready to go for deer, but did not have the strength to leave his grandson alone, and now that the boy seemed a little better, the old man decided to try his luck. The old Malanya agreed to look after the boy, too --- it was all better than if the boy layed in the hut all by himself.

Yemelya felt right at home in the woods. How could he not when he spent his whole life wandering around it with his rifle and dog. All paths, all signs [hallmarks, postmarks] --- the old many new them all for a hundred verst [versta -- an equivalent of 1066.781 meters] around.

And now, at the end of June, it was especially nice in the woods: the grass was dappled with beautiful flowers, just opened, the air was full with an amazing fragrance of the herbs, and the tender summer sun was looking down enveloping the trees, and the grasses, and a creek splashing in the sedge [osoka], and the far mountains with its bright light.

Yes, everything was good and well all around, and Yemelya stopped more than once to catch his breath and to look around.

The twisting path he was following was climbing the mountain past the big rocks and the sharp drops. All the big trees were cut down here, but young birches and bushes of honeysuckle nestled by the roadside,  interspersed with rowanberry's lush green canopy.
% this one is a difficult sentence
Here and there you'd see dense copses of young firs that would stand on the side of the path like a harsh brush, their large bushy branches merrily bristling about.
One location, somewhere halfway up the mountain, offered a wide [all-encompassing] view of the surrounding peaks and Tichki. The village was almost invisible at the bottom of the deep mountain trough, and peasant huts seems like black dots from up here. Shielding his eyes from the sun, Yemelya stood there for a long time looking at his hut and thinking about his grandson.

``Well, Lisko, hark\ldots'' said Yemelya when they have made it down the other side of the mountain and left the path for the blindly thick fir grove.

You did not have to tell Lisko twice. He knew his craft well so, burying his pointed face in the ground, he disappeared into the dense green thicket. Only once his back with characteristic yellow spots showed.

The hunt began.

The giant firs rose up to the sky with their sharp peaks. Bushy branches had grown into each other forming an impenetrable dark ceiling [svod] about the hunter's head, and only in a few places the bright gay ray  would cut through and burnish the yellowish moss or a wide fern leaf with its golden light. The grass does not grow in groves like this, and so Yemelya was walking on the yellowy moss, soft as a carpet.

The hunter was plodding through this grove for hours. Lisko disappeared without a trace. The only sounds were a crack of a broken branch under the foot or a motley [?] woodpecker flying from on thee to the next. Yemelya was scanning all around him for traces of deer: if only some sign somewhere, be it the branches that the deer broke with his antlers, or his cloven hoof imprinted in the moss, or the grass on the hillocks all eaten up. It started to get dark. The old man was feeling tired. It was time to think about shelter for the night.

``Must be that other hunters have scared away the deer.'' --- Yemelya was thinking.

But here came a muffled cry from Lisko, and the branches started crackling [crack] somewhere ahead. Yemelya leaned into a fir and started waiting.

This was a stag. A real ten antler beautiful deer, the most noble of the forest citizens. Here he reclined his branching antlers all the way to his back and was listening intently, sniffing the air, only to fly into the green thicket the very next moment.

Old Yemelya saw the deer, but was too far from it: the bullet would not reach there. Lisko was laying low in the thicket, frozen to the ground and waiting for the gunshot; he hears the dear, feels its odor\ldots Here broke the shot, and the deer launched forward like an arrow. Yemelya missed, and Lisko howled in his desperate hunger. The poor dog already was dreaming of the smell of the roasting deer, he saw an appetizing bone that the master would toss him, instead he had to sleep with an empty belly tonight. A  tragic turn of events indeed\ldots

``Well, let that deer run freely some more,'' --- Yemelya was thinking aloud while sitting near the fire by the bushy hundred-year-old fir that night. ``We need to get ourselves a little fawn, Lisko\ldots You know?''

The dog would only wags his tail in misery, his pointed face settled between his front paws. In the whole day, he barely had a single slice of dried bread thrown to him by Yemelya.

%%%%%%%%%%%%%%%%
% Part 3
%%%%%%%%%%%%%%%%
\section{}

Yemelya prowled on for three days in the woods with Lisko, and all in vain: he did not chance upon a doe with a fawn. The old man felt that his stamina was at an end, but he did not have the heart to come back home emptyhanded. Lisko was also depressed and got even thinner, despite having snacked on a couple of young rabbits.

They had to find shelter for a third night by the fire in the woods. But even in his sleep Yemelya was dreaming of the yellow fawn Greeshook was asking for; the old man would track the prey for a long time, ready the rifle, but the deer would escape right from under his nose. Lisko, it seemed, dreamed of deer too --- he would yelp in his sleep and bark under his breath.

It was only on the fourth day, when both the man and the dog were on the verge of exhaustion, that they discovered a trace of a doe with a fawn by mere good luck. It happened in a thick new fir growth on a side of a mountain. At first, Lisko found the place where the deer slept overnight, and then he followed its tangled tracks in the grass.

``Female with a fawn,'' --- thought Yemelya observing prints of pairs of big and small hooves in the grass. ``They were here this morning\ldots Lisko, search, dear!..''

The day was hot. The sun beat down mercilessly. The dog sniffed around the bushes and in the grass with his tongue out; Yemelya could barely move his legs. But here came a familiar sound of cracking and rustling\ldots Lisko dropped into the grass and froze. Grandson's words kept beating in Yemelya's head: ``Gramps, get a fawn\ldots and it needs to be yellow.'' There was the doe\ldots She was a gorgeous [stately] female. She stood on the edge of the woods and warily stared right at Yemelya. A swarm of buzzing insects were whirling above the deer and made her shudder.

``No, you won't fool me\ldots'' Yemelya thought, slowly moving from his hiding place.

The deer had long sensed the hunter, but was bravely watching his movements.

``This would be the female trying to lead me away from the fawn,'' Yemelya thought while inching on.

When the old man was about to take aim, the deer carefully treaded several yards away and stopped. Yemelya still inched on with his rifle. Again and again he would slowly move closer to the deer, and the deer would disappear as soon as Yemelya would cock the gun.

``You won't stray away from the calf.'' Yemelya repeatedly whispered while patiently tracking the animal over many hours.

This struggle between the man and the beast lasted until the very nightfall. The noble animal risked its life ten times trying to lead the hunter away from the hidden calf; old Yemelya was both angry with her and amazed at his victim's courage. There was no chance she could escape him\ldots Many times before he had to kill a sacrificing mother just like this. Lisko, like a shadow, crouched after his master, and when the man completely lost the sight of the deer, the dog nudged him with his hot snout. The old man turned around and sat. Twenty yards away from him, under the honeysuckle bush, stood that very same yellow calf that he chased after for three whole days. This was the prettiest fawn, just several weeks old, with yellow fluff and thin legs; his handsome head was thrown back, and his thin neck would stretch as he was trying to reach the branches above him. With a sinking [could be reverence, could be anticipation] heart, the hunter pulled back the hammer on the rifle and aimed at the young, defenseless animal's head\ldots

Another second --- and the little calf would have fallen down into the grass with a plaintive dying cry; but right at that moment the old hunter remembered how heroically the calf's mother was trying to protect him, he remembered how his Greeshootka's mother saved him from the wolves by giving up her life. As if a thread snapped in Yemelya's chest and he lowered the gun. The calf still walked around the bush, plucking the leaves and listening for the slightest rustle. Yemelya rose and whistled -- the little animal disappeared into the bushes with a lightning speed.

``Hah, what a runner'' said the old man, wistfully smiling at his own thoughts. ``Did not waste any time: like an arrow\ldots He ran away, Lisko, didn't he, our calf? Well, such a runner, he needs to grow up a little\ldots Ah, what a nimble creature!''

The old man stood in one place for a long time, still smiling and thinking of the agile runner.

The next day Yemelya was nearing his hut.

``Ah\ldots gramps, did you ge the calf?'' -- Greesha met him after waiting for the old man this whole time with anticipation.

``No, Greeshootka\ldots but saw him\ldots''

``Was he yellow?''

``All yellow, but with a black nozzle. Stood under a bush plucking the leaves\ldots I aimed\ldots''

``And missed?''

``No, Greeshook; I took pity on the little guy\ldots Took pity on his mother\ldots I whistled, and the calf --- he took right off into the thicket, and that was it. He ran away, the little urchin.''

The old man took a long time telling Greeshootka how he looked for the calf for three days and how the calf ran away from him. The boy listened and merrily laughed with his old gramps.

``But I did bring you a capercaillie, Greeshook'' added Yemelya after finishing the story. ``The wolves would've eaten him anyway.''

Capercaillie's feathers were plucked, and he ended up in a pot. The sick boy ate the soup with pleasure and later, when falling asleep, just kept asking his gramps:

``So he ran away, the calf?''

``He did, Greeshook.''

``All yellow?''

``All yellow, and just the nozzle was black and his little hooves.''

The boy fell asleep and, throughout the night, he kept dreaming of a little yellow calf walking merrily in the woods with his mother. The old man slept on the furnace and, like his grandson, kept smiling in his sleep.

\end{document}
